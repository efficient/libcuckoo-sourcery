\documentclass[12pt, letterpaper]{article}
\usepackage{graphicx}
\usepackage{subcaption}
\usepackage{fullpage}

\usepackage{lmodern}
\usepackage[T1]{fontenc}
\usepackage{textcomp}


\title{Benchmarking Libcuckoo}
\author{Manu Goyal}
\date{\today}

\newcommand{\myfigwidth}{0.8\textwidth}

\begin{document}

\maketitle

In this benchmark, we compare libcuckoo, our high-performance
concurrent hash table, with the Intel Thread Building Blocks
{concurrent\textunderscore hash\textunderscore map}, a popular
concurrent hash table, and the STL {unordered\textunderscore map}, a
single-threaded hash table. In each benchmark, we look at each
hashtable's throughput with integer keys and string keys. The machine
we ran the benchmarks on has 16 CPU's split into two NUMA clusters and
64GB memory. It ran Ubuntu 12.04, and the benchmarks were compiled
with \texttt{g++-4.8}. We ran the benchmarks with all 16 CPU's, except
for STL, which we had to run on only one core since it is a
single-threaded hash table. In all the benchmarks, libcuckoo and TBB
significantly outperformed STL as expected, so we focus on comparing
libcuckoo and TBB.

\section{Inserts}
\label{sec:inserts}

Our insert benchmark measures the time taken to fill up a table from
0\% to 90\% of its allocated capacity. Figure~\ref{fig:inserts} shows
the insert throughput of the three tables with integer and string
keys. libcuckoo significantly outperforms TBB and STL on both integers
and strings, producing about 85\% and 71\% more throughput on integers
and strings, respectively. We suspected that TBB's low performance was
due to the fact that it doesn't deal well with multiple NUMA clusters.
On one NUMA cluster (8 CPU's), TBB does significantly better, but
libcuckoo still outperforms it by about 55\% and 31\% on integers and
strings, respectively.

\begin{figure}
  \centering
  \includegraphics[width=\myfigwidth]{inserts_plot}
  \caption{Insert throughput}
  \label{fig:inserts}
\end{figure}

\section{Reads}
\label{sec:reads}

Our read benchmarks fills a table up to 90\% of its allocated
capacity, then concurrently runs reads for data that is in the table
as well as data that isn't in the table. It counts the number of reads
executed over 10 seconds. Figure~\ref{fig:reads} shows the read
throughput of the three tables with integer and string keys. For
integer reads, libcuckoo and TBB were fairly close, but TBB ended up
outperforming libcuckoo by about 15\% for the largest table size. For
strings, libcuckoo did better, outperforming TBB by about 34\%.

\begin{figure}
  \centering
  \includegraphics[width=\myfigwidth]{reads_plot}
  \caption{Read throughput}
  \label{fig:reads}
\end{figure}

\section{Mixed}
\label{sec:mixed}

Our mixed benchmark runs a mixed workload of inserts and reads at a
configurable ratio, and measures the time and number of operations
taken to fill up the table from 0\% to 90\% of its allocated capacity.
Figure~\ref{fig:mixed} shows the mixed throughput of the three tables
with integer and string keys. As the percentage of inserts increases,
libcuckoo does successively better than TBB, outperforming it by 15\%
at 10\% inserts and 82\% at 90\% inserts for integers.

\begin{figure}
  \centering
  \includegraphics[width=\myfigwidth]{mixed_plot}
  \caption{Mixed throughput}
  \label{fig:mixed}
\end{figure}

\section{Memory}
\label{sec:memory}

While not a completely accurate measure of the amount of memory used
by each table, we measured the maximum resident set size as determined
by Ubuntu's \texttt{time} command for the read benchmark.
Figure~\ref{fig:memory} shows the approximate memory usage of the
three tables with integer and string keys. For integers, libcuckoo
scales far better than TBB, using 33\% as much memory as TBB with the
largest table. For strings, since the memory used by the strings
themselves dominates the memory usage, there is a less significant
difference between libcuckoo and TBB.

\begin{figure}
  \centering
  \includegraphics[width=\myfigwidth]{memory_plot}
  \caption{Memory usage}
  \label{fig:memory}
\end{figure}

\section{Conclusion}
\label{sec:conclusion}

libcuckoo has a number of features that cause it to perform better
than TBB and STL in a majority of cases. libcuckoo stores data in a
cache-optimized form and avoids false sharing between CPU's, which
allow it to scale very well to a large number of CPU's while still
using very little extra memory. Additionally, libcuckoo implements
partial-key hashing to reduce the comparison time for expensive data
types like strings. While TBB does do a better job on a pure read
workload of integers on a large table, we believe that for most use
cases, libcuckoo is a better choice than both TBB's
{concurrent\textunderscore hash\textunderscore map} and the STL
{unordered\textunderscore map}.

\end{document}
